\chapter{Introduction}\label{ch:intro}
%these sections are optional, up-to the author
\section{Motivation}
In the modern world, digital technology is one of the main factors of competitiveness. It has changed the way people work, consume and communicate. Technological revolutions are taking place in all spheres of life: medicine, education, business, etc. The need to introduce information technology solutions for charity is also one of the topical issues.

According to the data, there are 35 official funds in Kazakhstan \cite{Egov}. But there are also many charitable foundations that are not mentioned in the official website. There are a lot of funds, but why don't people know about them? 

Charity is the provision of gratuitous assistance or donations of funds with the motivation to help other people or animals. In simple words, a good deed is based on charity. Since our project focuses on charity and assistance, we began to study about the state of charity in Kazakhstan. \cite{lawRk} The result showed us that charity in the Kazakh culture has some gaps. Many who want to help – do not know how. Often people don't trust charities. Why? Because not all funds publish reports on what the funds were spent on. Since it is really difficult to track what money is being spent on. Fraud and money laundering have become seen as the basis of charitable activities. In big cities there are many fundraising points, there are some events about good deeds, but there is no information about them anywhere if it were not extensive. Why is there no single solution for all these problems?

\newpage
\section{Aims and objectives}
The main goal we were considering was that there is no centralized application in our economic environment at the moment. The main thing to pay attention to is that our application solves such global problems, as in the example of people who refuse charity because of distrust of various funds and fees. And your project is related to the registration of funds and fees according to a certain template, where they must provide the appropriate documents to identify it.

If we take a closer look at how everything is implemented at the moment, then we can understand that in Kazakhstan many donations go through grapevine. That is, as we learned earlier, we do not have a centralized application as such, and this is very difficult. After all, grapevine is not always safe and truthful, because now there are many scammers who work according to this scheme, who do not provide any supporting documents, or any substantial evidence.

Our main task was so that we could make a convenient application where, as mentioned earlier, everything was centralized. That is, an application that is convenient to use for different segments of the population of our country, where not only individual fundraising for various diseases and so on will be concentrated, but also various funds and collection points for those in material need. And we have made preferences in the direction of mobile applications, since at the moment, this is the side of development that is very popular and used. We made the site only for a part of the admin, that is, for the management staff, since by making our project in the form of a website, we would have lost a significant number of our users, since now many operations and everyday tasks are done from mobile devices.




% The first chapter is \nameref{ch:intro} chapter. It is this one that you are currently reading. It gives insight into the work done. In Chapter \ref{ch:A} we review related work and formulate the problem to solve. Chapter \ref{ch:B} is describing the solution to the problem. And in \nameref{ch:concl} chapter we conclude our conclusion.